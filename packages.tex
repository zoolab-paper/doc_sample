
% customized
\usepackage[utf8]{inputenc}
\usepackage{xspace}
\usepackage{xcolor}
\usepackage{graphicx}

\usepackage{pifont}
\usepackage{amsmath}
\usepackage{amsfonts}
\let\Bbbk\relax         % avoid amssymb conflicting with other modern pkgs
\usepackage{amssymb}    % for \varnothing
\usepackage{algorithm}
\usepackage{algorithmicx}
\usepackage{algpseudocode}
\usepackage{enumitem}
\usepackage{tikz}
\usepackage{url}

\usepackage{graphicx}
\graphicspath{{images/}{figures/}}

%%% choose from either subfigure, or caption + subcaption
%\usepackage{subfigure}
%% \begin{figure}
%%   \subfigure[caption]{
%       \include ...
%       \label
%    }
%% \end{figure}
\usepackage{caption}
\usepackage{subcaption}
%% \begin{figure}
%%   \begin{subfigure}[]{.8\columnwidth}
%%   \include ...
%%   \label ...
%%   \caption{...}
%%   \end{subfigure}
%% \end{figure}

% NEW coding style
%% need python pygments package: brew install pygments
\usepackage{minted}
\usemintedstyle{manni}
\renewcommand\theFancyVerbLine{\footnotesize\arabic{FancyVerbLine}}
\setminted{
    mathescape,
    frame=none,
    framesep=2mm,
    fontsize=\footnotesize,
    linenos,
    numbersep=5pt,
    numbers=left,
    tabsize=2,
    autogobble,
    xleftmargin=0.4cm,
}

% OLD coding style
\usepackage{listings}
\usepackage{color}
\lstset{
  basicstyle=\footnotesize\ttfamily,
  frame=none,	% none, single
  numbers=left,
  language=C,
  xleftmargin=0.4cm,
  showstringspaces=false,
  numbersep=5pt,
  numberstyle=\footnotesize\color{black},
  tabsize=2
}
% xleftmargin=0.25in

%% tikz: circled texts
\newcommand*\circled[1]{\tikz[baseline=(char.base)]{\node[shape=circle,draw,inner sep=1pt] (char) {\scriptsize #1};}}

%% custom editing commands
\newcommand{\todo}[1]{\textcolor{red}{[TODO: #1]}}

%% system name
\def\sys{SysName\xspace}

%% chinese support:
%%%% 1. only work with xeLaTeX (instead of pdfLaTeX)
%%%% 2. does not work with the USENIX template
%%%% 3. simply type UTF-8 traditional chinese characters in the document
%%%% 4. to build, use 'latexmk -pdfxe'
\iffalse	% iffalse or iftrue
\usepackage[BoldFont, SlantFont, CJKchecksingle, CJKmath=true]{xeCJK}
\setCJKmainfont{AR PL KaitiM Big5}
\setCJKsansfont{AR PL Mingti2L Big5}
\XeTeXlinebreaklocale "zh"
\XeTeXlinebreakskip = 0pt plus 1pt
\fi

